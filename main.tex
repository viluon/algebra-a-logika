\documentclass[a4paper]{article}

\usepackage[utf8]{inputenc}
\usepackage[T1]{fontenc}
\usepackage{textcomp}
\usepackage[english]{babel}
\usepackage{amsmath, amssymb, bm}
\usepackage{tcolorbox}
\usepackage[makeroom]{cancel}
\usepackage{csquotes}

% figure support
\usepackage{import}
\usepackage{xifthen}
\pdfminorversion=7
\usepackage{pdfpages}
\usepackage{transparent}
\newcommand{\incfig}[1]{%
	\def\svgwidth{\columnwidth}
	\import{./figures/}{#1.pdf_tex}
}

% style for tcolorboxes
\tcbset{plain/.style={colbacktitle=white,coltitle=black,colback=white}}
\pdfsuppresswarningpagegroup=1

\title{BI-ALO notes}
\begin{document}
	\maketitle

	\section{Doporučená literatura}
	\begin{itemize}
		\item P. Štěpánek: Matematická logika (MFF)
		\item A. Sochor: Klasická M. L. (UK, 2003)
		\item J. Barwise: Handbook of M. L.
		\item F. Hausdorff: Mengerlehne (1914)
		\item Course pages - ALO: matematická logika (PDF)
	\end{itemize}

	\section{Osnova}
	\begin{enumerate}
		\item Úplnost predikátové logiky
		\item Rezoluce pro predikátovou logiku
		\item Úvod do teorie množin
	\end{enumerate}

	\section{Formální systém pro predikátovou logiku}
	\subsection{Redukce jazyka}
	$\neg, \to,  \forall$ ("$(\exists x) \varphi$" je zkratka za "$\neg (\forall x) \neg \varphi$")

	\subsection{Axiomy}
	Pro každé formule $A, B, C$ jazyka L jsou následující formule axiomy:
	\begin{description}
		\item[H1] $A \to (B \to A)$
		\item[H2] $(A \to (B \to C)) \to ((A \to B) \to (A \to C))$
		\item[H3] $(\neg B \to \neg A) \to (A \to B)$
		\item[S] Pro každou formuli $\varphi$, proměnnou $x$ a term $t$ (substituovatelný za $x$ do $\varphi$)
			je následující formule axiom: $(\forall x) \varphi \to \varphi_x[t]$ ("schéma axiomu specifikace)
		\item[P] Pro každé formule $\varphi, \psi$ a proměnnou $x$ která není volná ve fli $\varphi$ je
			následující fle axiom:  $(\forall x)(\varphi \to \psi) \to (\varphi \to (\forall x) \psi)$
	\end{description}

	\subsection{Odvozovací pravidla}
	\begin{description}
		\item[MP (modus ponens)] Z formulí $\varphi, \varphi \to \psi$ odvoď formuli $\psi$
		\item[G (generalizace)] Z formule $\varphi$ odvoď formuli $(\forall x) \varphi$
	\end{description}

	
	\begin{description}
		\item[Domácí úkol] napište překladač z prefixu do infixu (a naopak)

		\item[Důkaz] Konečná posloupnost formulí $\varphi_1, \varphi_2, \ldots, \varphi_n \equiv \varphi$ je
			\textit{důkaz} (v teorii $T$), pokud každá z formulí $\varphi_i, i \le n$ je buďto
			axiom, nebo formule z teorie $T$, nebo je z nějakých předchozích odvozena pomocí některého
			z odvozovacích pravidel.

			Pokud formule $\varphi$ má nějaký takový důkaz, řekneme, že je \textit{dokazatelná}
			(\textit{dokazatelná v teorii $T$}), psáno $\vdash \varphi$, resp. $T \vdash \varphi$.

		\item[Domácí úkol] napište proof checker, který zkontroluje důkaz podle
			definice výše.

		\item[Teorie] Libovolná množina formulí v nějakém jazyce $L$ se nazývá \textit{teorie}.
			\begin{itemize}
				\item Teorie grup v jazyce $+, -, 0$:
					\begin{enumerate}
						\item $(\forall x)(\forall y)(\forall z) ((x + y) + z = x + (y + z))$
						\item $(\forall x) (x + 0 = x = 0 + x)$ 
						\item $(\forall x) (x + (-x) = 0)$
					\end{enumerate}
					Modely této teorie se nazývají "grupy:"
					$(\mathbb{Z}, +, -, 0), (\mathbb{R}^+, +, \cdot^{-1}, 1)$
				\item Teorie uspořádání v jazyce $<$
					 \begin{enumerate}
						 \item $(\forall x) \neg (x < x)$
						 \item $(\forall x)(\forall y)(\forall z) (x < y \land y < z \to x < z)$
					\end{enumerate}
					Modely této teorie jsou "uspořádané množiny:"
					$(\mathbb{R}, <), (P(\mathbb{N}), \subset), \ldots$
			\end{itemize}
		\item[Věta o korektnosti] Buď $T$ teorie v jazyce $L$ a $\varphi$ nějaká formule $L$.
			Potom pokud je $\varphi$ dokazatelná v teorii $T$, pak platí v každém jejím modelu.
			Jinými slovy, $(T \vdash \varphi) \to (T \models \varphi)$.
			\begin{description}
				\item[Důkaz] Buď $\varphi_1, \varphi_2, \ldots, \varphi_n \equiv \varphi$ formální důkaz formule
					$\varphi$ v teorii $T$. Buď $M \models T$ lib. model.
					Indukcí pro $\varphi_i, i \le n$ ukážeme, že $\varphi_i$ platí v $M$ při
					každém ohodnocení.
					\begin{enumerate}
						\item Pro formuli $\varphi_i \in T$, je $M \models \varphi_i$ z definice. 
						\item Je-li $\varphi_i$ axiomem výrokové logiky, tak je to tautologie,
							takže platí (dokonce) v každé realizaci jazyka $M \models L$.
						\item Je-li $\varphi_i$ axiomem specifikace tvaru
							$((\forall x) \psi) \to \psi_x[t]$: buď BÚNO
							$M \models (\forall x) \psi$. To znamená, že $M \models \psi[e(x/m)]$ 
							pro každé ohodnocení $e$ a každé $m \in M$, speciálně pro
							$t[e] \in M$. To ale znamená, že $M \models \psi[e(x/t)]$,
							neboli $M \models (\psi_x[t])[e]$.
						\item Je-li $\varphi_i$ axiom tvaru
							$(\forall x)(\psi \to V) \to (\psi \to (\forall x)V)$,
							kde $x$ není volná v $\psi$: buď $M \models (\forall x)(\psi \to V)$.
							To znamená,, že pro každé $m \in M$ je
							$M \models (\psi \to V)[e(x/m)]$, tj. buďto
							$M \cancel{\models} \psi[e(x/m)]$ nebo $M \models V[e(x/m)]$.
							Přitom pro $M \cancel{\models} \varphi[e(x/m)]$ je $M \models \varphi[e]$.
							Tedy buďto  $M \cancel{\models} \varphi[e]$ nebo $M \models (\forall x)V[e]$,
							každopádně $M \models (\varphi \to (\forall x)V)[e]$.
						\item Je-li $\varphi_i$ některý z \textit{axiomů rovnosti}, je dokonce
							$M \models \varphi_i$ pro každý $M \models L$.
						\item Pokud $\varphi_i$ vznikla z předchozích $\varphi_j, \varphi_j \to \varphi_i$ pomocí
							MP, přičemž pro fle $\varphi_j, \varphi_j \to \varphi_i$ už máme $M \models \varphi_j$,
							$M \models \varphi_j \to \varphi_i$ je z definice taktéž $M \models \varphi_i$ 
						\item Pokud $\varphi_i$ vznikla z nějaké předchozí $\varphi_j$ jako $(\forall x)\varphi_j$ 
							pomocí G, přičemž pro $\varphi_j$ už máme $M \models \varphi_j$, je
							$M \models \varphi_j[e]$ pro každé ohodnocení, tedy speciálně
							$M \models \varphi_j[e(x/m)]$ pro každé $m \in M$, to znamená
							právě $M \models (\forall x) \varphi_j[e]$.
					\end{enumerate}
				\item[Symetrická grupa] značena $S_M : \{\pi : M \to M | \pi \text{
					je permutace, tj. prostá a na}\}$, na příklad $S_{\{1, 2, 3\}}$
				\item[Příklad] formule $(\forall x)(\forall y)(x + y = y + x)$ \textit{není} dokazatelná
					v teorii grup. Například $(S_{\{1, 2, 3\}}, \circ, \cdot^{-1}, \text{id})$
					je nekomutativní grupa: \begin{align*}
						(1 3 2) \circ (3 1 2) &= (3 2 1) \\
						(3 1 2) \circ (1 3 2) &= (2 1 3)
					.\end{align*}
					Stejně tak ale teorie grup nedokazuje formuli
					$\neg (\forall x)(\forall y)(x + y = y + x)$.
					$(T \cancel{\vdash} \varphi) \land (T \cancel{\vdash} \neg \varphi)$.
			\end{description}
		\item[Věta o úplnosti] (Kurt Gödel, 1929; Henkin, 1947) Buď $L$ jazyk predikátové logiky,
			buď $T$ teorie v $L$, buď $\varphi$ formule jazyka $L$. Potom $T \vdash \varphi$ právě
			když $T \models \varphi$. (Směr zleva doprava je známý z \textit{věty o korektnosti}.)
		\item[Věta] Teorie je bezesporná právě když má model. (V této větě dokazuje
			věta o korektnosti směr zprava doleva.)
			\begin{description}
				\item[důkaz] Buď $T \models \varphi$; chceme ukázat, že $T \vdash \varphi$.
					Kdyby ne, pak $T, \neg \varphi$ je bezesporná, tedy má model
					$M \models T \cup \{\neg \varphi\}$. To je model $M \models T$,
					ve kterém $M \not \models \varphi$.
				\item[idea pro konstrukci modelu] Universum budou tvořit
					konstantní termy. Konstanty realizují samy sebe.
					Pro funkční symbol $f$ buď $f(t_1, \ldots, t_n)$ konstantní term.
					Pro relační symbol $R$ buď $M \models R(t_1, \ldots, t_n) \iff
					T \vdash R(t_1, \ldots, t_n)$. Tyto pravidla přináší několik
					problémů:
					\begin{enumerate}
						\item Pokud $L$ nemá konstanty, je tento \enquote{model} prázdný.
						\item Pro dva různé konstantní termy $s, t$ může být
							$T \vdash (s = t)$.
						\item Pro uzavřenou $\varphi$ musí být $M \models \varphi$ nebo
							$M \models \neg \varphi$. Přitom nemusí $T \vdash \varphi$
							ani $T \vdash \neg \varphi$.
						\item Pokud $T \vdash (\exists x) \varphi(x)$, potřebujeme
							$T \vdash \varphi_x[c]$.
					\end{enumerate}
			\end{description}
		\item[Definice] Teorie $T$ v jazyce $L$ je \textit{úplná}, pokud pro
			každou uzavřenou formuli $\varphi$ jazyka $L$ je buďto $T \vdash \varphi$
			nebo $T \vdash \neg \varphi$. Jinak je $T$ \textit{neúplná} a
			$\varphi$ je \textit{nerozhodnutelná}. \textbf{Úplná teorie
			je bezesporná}.
			\begin{description}
				\item[příklady] \enquote{Typická} teorie je neúplná.
					\begin{enumerate}
						\item Teorie grup je neúplná: nerozhoduje sentenci
							$(\forall x)(\forall y)(x + y = y + x)$
						\item Teorie hustých lineárních neomezených uspořádání
							je úplná (\enquote{ve všech modelech platí to, co v $\mathbb{R}$})
						\item $Th(W, +, *, 0, 1)$, nebo obecně
							$Th(M) = \{\varphi : M \models \varphi \text{ uzavřená}\}$
					\end{enumerate}
			\end{description}
		\item[Definice] Teorie $T$ je \textit{Henkinovská}, pokud pro každou existenční
			sentenci tvaru $(\exists x) \varphi(x)$ existuje v jazyce teorie $T$
			nějaká konstanta $c$ taková, že
			$T \vdash ((\exists x) \varphi(x) \implies \varphi_x[c])$.
			(Henkinovské konstanty jsou \textit{svědci existence}.)

			Například teorie uspořádání v jazyce $\le$ \textit{není} Henkinovská:
			nemá svědčící konstantu pro $(\exists x)(\forall y)(x \le y)$.
			Můžeme do jazyka přidat konstantu \enquote{$0$} a do teorie
			přidat axiom $(\exists x)(\forall y)(x \le y) \implies (\forall y)(0 \le y)$.
			(Ta \textit{stále není Henkinovská}.)
		\item[Věta (Henkin)] Každá úplná bezesporná Henkinovská teorie \textit{má} model.
			\begin{description}
				\item[důkaz] Buď $\tau$ množina konstantních termů jazyka $L$ teorie $T$.
					Pro term $t \in \tau$ buď $[t] = \{s \in \tau : T \vdash s = t\}$ jeho
					\enquote{ekvivalenční třída.} Nosnou množinou $M$ (neboli \textit{universem})
					hledaného modelu $\mathbb{M}$ bude právě množina těchto ekvivalenčních tříd
					$[t]$.

					Realizace jazyka $L$:
					\begin{itemize}
						\item Pro konstantu $\le$ jazyka $L$ je $c^{\mathbb{M}}$ prvek
							$[c] \in M$.
						\item pro n-ární funkční symbol $f$ a prvky
							$[t_1], [t_2], \ldots, [t_n] \in M$ buď
							$f^{\mathbb{M}}\left( [t_1], \ldots, [t_n] \right)$
							prvek $[f(t_1, \ldots, t_n)] \in M$.
						\item pro n-ární predikát $R$ a prvky $[t_1], \ldots, [t_n] \in M$ 
							buď $R^{\mathbb{M}}([t_1], \ldots, [t_n]) \iff
							T \vdash R(t_1, \ldots, t_n)$
					\end{itemize}
				\item[Fakt] Definice výše nezávisí na reprezentantech (resp. volbě reprezentantů).
					Tj. pro $[s_i] = [t_i]$, tj. $s_i \in [t_i]$, tj $T \vdash s_i = t_i$ 
					je $f^{\mathbb{M}}([s_1], \ldots, [s_n])$ individuum $[f(s_1, \ldots, s_n)]$,
					což je tentýž objekt jako $[f(t_1, \ldots, t_n)]$.
				\item[důkaz - pokračování] Ukážeme, že tato struktura
					$\mathbb{M} =
					(M, c^{\mathbb{M}}, \ldots, f^{\mathbb{M}}, \ldots, R^{\mathbb{M}}, \ldots)$
					\textit{je} modelem $T$, tj. že každý axiom teorie $T$ je v $\mathbb{M}$ 
					splněn. Ve skutečnosti ukážeme, že pro každou uzavřenou formuli
					$\varphi$ jazyka $L$ je $\mathbb{M} \models \varphi \iff T \vdash \varphi$
					(indukcí).
					\begin{enumerate}
						\item Pro $\varphi$ tvaru $t_1 = t_2$ je $\mathbb{M} \models \varphi$
							právě když $[t_1]$ a $[t_2]$ je stejný prvek nosné
							množiny $M$. To je právě když $T \vdash (t_1 = t_2)$.
						\item Pro $\varphi$ tvaru $R(t_1, \ldots, t_n)$ je
							$\mathbb{M} \models \varphi$ právě když $T \vdash R(t_1, \ldots, t_n)$.
							($T$ je bezesporná!)
						\item Pro $\varphi$ tvaru $\neg \psi$, kde pro $\psi$ je
							již tvrzení dokázáno:
							\begin{align*}
								&\mathbb{M} \models \varphi \\
								\iff& \mathbb{M} \models \neg \psi \\
								\iff& \mathbb{M} \not \models \psi \\
								\iff& T \not \vdash \psi \\
								\iff& T \vdash \neg \psi
							.\end{align*}
						\item Pro $\varphi$ tvaru $\psi \implies \vartheta$, tj.
							$\mathbb{M} \models (\psi \implies \vartheta)$, právě když
							$\mathbb{M} \not \models \psi$ nebo $\mathbb{M} \models \vartheta$,
							tj. právě když $T \not \vdash \psi$ (čili $T \vdash \neg \psi$)
							nebo $T \vdash \vartheta$, to je právě když
							$T \vdash (\psi \implies \vartheta)$:
							\begin{enumerate}
								\item Je-li $T \vdash \neg \psi$, použijeme
									$\vdash \neg \psi \implies (\psi \implies \vartheta)$ a MP
									dává $T \vdash \psi \implies \vartheta$
								\item Je-li $T \vdash \psi$, použijeme H1
									$\vartheta \implies (\psi \implies \vartheta)$ a MP
									dává $T \vdash \psi \implies \vartheta$
							\end{enumerate}
						\item Pro $\varphi$ tvaru $(\forall x) \psi$, přičemž pro všechny
							instance formule  $\psi$ už je tvrzení dokázáno.
							Z úplnosti je buďto $T \vdash \varphi$ nebo $T \vdash \neg \varphi$.
							\begin{enumerate}
								\item Je-li $T \vdash \varphi$, tj. $T \vdash (\forall x) \psi$,
								použijeme axiom S a dostaneme $T \vdash \psi_x[t]$ pro každý term,
								tj. $\mathbb{M} \models \psi[e(x/[t])]$ pro každé $[t] \in M$,
								tj. $\mathbb{M} \models (\forall x) \psi$.
								\item Je-li naopak $T \vdash \neg \varphi$, tj.
									$T \vdash \neg (\forall x) \psi$, tj.
									$T \vdash (\exists x) \neg \psi$, (\textbf{Henkin}) tedy
									máme konstantu $c \in L$ takovou, že
									$T \vdash ((\exists x) \neg \psi \implies \neg \psi_x[c])$,
									MP dává $T \vdash \neg \psi_x[c]$, čili
									$\mathbb{M} \models (\exists x) \neg \psi$, tj.
									$\mathbb{M} \models \neg (\forall x) \psi$.
							\end{enumerate}
					\end{enumerate}
				\item[domácí úkol] \enquote{free algebra,} např. volný monoid $A*$ slov
					nad abecedou $A$ s operací zřetězení a neutrálním prvkem $\varepsilon$.
			\end{description}
	\end{description}


\end{document}
